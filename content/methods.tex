Studying the dynamics of solid-state systems far from equilibrium with ultrafast experimental techniques has gained attraction over the last decades.
These techniques allow to probe different observables on femtosecond timescales, shedding light on the relaxation pathways towards equilibrium.

A simple concept to reach high time resolutions is the pump-prope technique.
The basic idea is that an excitation event (pump) and a probe event with a known variable time difference occur on the sample under investigation.
The excitation event may be an arriving optical laser pulse and the probe event might be the scattering of an electron or X-ray pulse.
If probe-events are measured while continiously varying the time delay the data can be assembled into a \emph{movie}.
Both, excitation and probe event, are required to be shorter than the time scale of the pheonomenon under investigation.
If the phenomenon under study is reversable and there is enough time between each excitation event, a measurement can be done continiously on one sample.
Current plans do not involve studying irreversible phenomena.


%\[ I_1(\mathbf{q}) \propto \sum_\mathrm{i} \frac{n_\mathrm{i}(\mathbf{k})+\frac{1}{2}}{\omega_\mathrm{i}(\mathbf{k})}\,\left| \sum_\mathrm{s} \e^{-W_\mathrm{s}(\mathbf{q})} \frac{f_\mathrm{s}(\mathbf{q})}{\sqrt{\mu_\mathrm{s}}} \left( \mathbf{q}\cdot\mathbf{e}_\mathrm{i, s}(\mathbf{k}) \right) \right|^2 \]

%capabilities of the machine
%description and discussion of observable
%what can you learn from tr-phonons
%epc
%samples are easy to use
%peak width <-> correlation length
%single phonon structure factor calculation
%multi-phonon structure calculation