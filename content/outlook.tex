\section{Proposed Research and Timeline}
\textbf{Diffraction on \ts}

Electron diffraction experiments done in trasmission naturally require the samples to be transparent for electrons.
Methods of preparing thin samples of layered materials like \acp{TMD} include exfoliation\cite{exf} and microtomy\cite{micro}.
Both methods are established and available in the Sample Preparation Lab of McGill's Physics Department.
Due to the layered structure the samples' surfaces will be parallel to the monolayers.
When conducting experiments with the electron bunches arriving normal to \ts\space samples and it's layers, the $\Gamma\mathrm{MK}$ plane of the \ac{BZ} is probed.
In order to probe the L points of \ts\space the sample needs to be tilted by 27\textdegree.
However, tilting the sample will decrease the time-resolution of the instrument, because electron and light pulse travel at different speeds.
For a sample with diameter of XXX that is tilted 27\textdegree and electrons travelling at half the speed of light the deviation of the probed time delay will be XXX.
To compensate the velocity mismatch the optical pump pulse can be tilted in order to probe the whole sample area at the same time delay\cite{baum2006}.