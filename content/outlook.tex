\section{Proposed Research and Timeline}
To start the research for my PhD thesis I seek to explore phonon softening and charge ordering phenomena in \ts.
Former work of the Siwick Group on \ts\space shows that photoexciting carriers into the 3d conduction band leads to a stiffening of transverse phonons at the M points\cite{otto2021}.
The coupling between free carriers and phonons, however, ist not enhanced.
This is in striking contrast to the behaviour observed in graphite\cite{stern2018}.
Both experiment were conducted at room temperature.
The \ac{CDW} phase with its excitonic ground state should be fragile against large densities of free charge carries due to screening effects.
It would be interesting to investigate the dynamics of the excitonic ground state decaying upon photo-excitation and its subsequent reformation upon cooling, while systematically varying the pump-fluence.
At this point in time it is not possible to control the temperature of the sample.

The installation of a closed-cycle cryostat is currently ongoing at will allow access to sample temperatures of 4-XXX\,K.
In addition to that, the machine is subject to more upgrades.
First, we have reduced the physical length of the instrument, reducing the influence of space-charge effects of the electron bunch on the time resolution.
Second, a faraday cup has been designed and machined and is ready for installation to measure number of electrons in the bunches.
This allows for further characterization and better control of the instrument.
Third, a new detector camera (Dectris Quadro) has been installed and implemented in the laboratory software system.
Instead of a conventional CCD chip coupled to a scintillator, the detector is a \emph{hybrid pixel detector} consisting of bare silicon pixels that can directly count incoming electrons.
Signal-to-noise ratio \textbf{radically} improved and the read-out electronics are fast enough to capture diffraction images of every single electron bunch.
Finally, the Dectris Quadro is much less sensitive to scattered light from pump and probe beam.

The previous work performed on \ts\space in the Siwick Group combined with ne new capabilities of the modified experimental setup provide a solid base to investigate the \ac{CDW} phase transition in new ways.
At a later stage it could be promising to not only probe the $\Gamma\mathrm{MK}$ plane of the \ac{BZ}, but also access the L point.
The problem of forfeiting time resolution has been overcome by other groups encountering similar issues in time-resolved diffraction experiments.
The idea is to compensate the velocity mismatch by tilting the optical pump pulse in order to probe the whole sample area at the same time delay\cite{baum2006,zhou2013}.

% bogen schlagen zu SC
% hopefully present preliminary results in presentation